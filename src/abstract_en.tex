%!TEX root = ../main.tex

\begin{changemargin}{0.8cm}{0.8cm}

~\vfill{}

\section*{Abstract}
\vskip 0.5em

% Achieving effective multi-agent \ac{UAV} coordination without explicit communication, alongside robust single-agent navigation in complex, \ac{GNSS}-denied environments, presents significant research challenges.
% Therefore, developing robust algorithms capable of handling these demanding conditions is necessary.
% This thesis addresses these challenges through two primary objectives. 
% The first objective is to extend the \ac{RBL} algorithm for multi-agent coordination from two to three dimensions, introducing new rules for enhanced performance. 
% The second objective involves implementing and validating this 3D \ac{RBL} framework for single \ac{UAV} navigation in a real-world forest environment using \ac{LiDAR} sensing.

% The 2D \ac{RBL} algorithm was extended to 3D by building upon its foundational concepts and introducing new rules for vertical control. 
% For single-agent forest navigation, this 3D \ac{RBL} was integrated with \ac{LiDAR}-based perception, utilizing Point-LIO for state estimation and a voxel-based mapping approach. 
% The system was evaluated through multi-agent simulations and real-world flight experiments in a forest.

% Simulations demonstrated the successful extension of \ac{RBL} to 3D, showing improved multi-agent coordination efficiency, particularly with more agents. 
% Real-world experiments validated the effectiveness of the implemented system for autonomous point-to-point navigation in a forest, though challenges with mapping dynamic environmental elements like leaves were identified as a limitation.

% In summary, this thesis validates the modification and real-world use of the \ac{RBL} algorithm for 3D \ac{UAV} point-to-point navigation.
% While the need for enhanced mapping is needed, the results of this study significantly contribute to the development of more independent and adaptable \ac{UAV} systems.

% The study of autonomous \acp{UAV} has become a prominent sub-field of mobile robotics.


\ac{UAVs} are increasingly transitioning from open-sky operations to complex, \ac{GNSS}-denied environments, presenting significant challenges for multi-agent coordination and single-agent autonomous navigation. 
This thesis addresses these issues by first extending the \ac{RBL} algorithm from two to three dimensions for multi-\ac{UAV} communication-free coordination.
New rules were developed to manage vertical movement and enhance 3D coordination efficiency. 
Secondly, the thesis investigates the practical application of this 3D \ac{RBL} for single \ac{UAV} navigation in a challenging forest environment, utilizing onboard \ac{LiDAR} sensing.

The methodology involved adapting the core \ac{RBL} for 3D space and integrating it with real-time perception systems. 
For the single-agent navigation task, this included \ac{LiDAR}-based point cloud processing, state estimation using Point-LIO, and voxel-based environmental mapping with Bonxai. 
Strategies to minimize limitations from anisotropic sensors, such as restricted \ac{LiDAR} \ac{FOV}, were also developed. 
The proposed algorithm was evaluated through extensive simulations in the multi-agent scenario and validated with real-world experiments in a forest environment for the single-agent case.
% The extended 3D \ac{RBL} was evaluated in simulated multi-agent coordination scenarios, while the integrated single-agent system was validated through real-world flight experiments in a forest.

Simulations confirmed the successful 3D extension of the \ac{RBL} algorithm, demonstrating effective multi-agent coordination. 
Real-world experiments validated the capability of the 3D \ac{RBL}-based system for autonomous point-to-point navigation in a forest. %, as illustrated in accompanying videos \cite{aggressive_flight}, \cite{conservative_flight}. 
However, limitations were identified, particularly in the mapping system's handling of dynamic environmental elements, which occasionally led to navigation failures.% \cite{flight_fail}.

% In conclusion, this thesis demonstrates the successful adaptation and practical application of the \ac{RBL} algorithm for 3D \ac{UAV} operations. 
% It contributes a 3D extension for multi-agent coordination and a validated system for single-agent navigation in complex, real-world settings, addressing challenges posed by sensor limitations. 
% While further work on robust mapping in dynamic environments is indicated, the findings provide a valuable foundation for advancing autonomous UAV systems.






\vskip 1em

{\bf Keywords} \Keywords

\vskip 2.5cm

\end{changemargin}
