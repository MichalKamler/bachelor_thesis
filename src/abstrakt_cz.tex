    %!TEX root = ../main.tex

\begin{changemargin}{0.8cm}{0.8cm}

~\vfill{}

\section*{Abstrakt}
\vskip 0.5em

\sloppy
% Výzkum na poli autonomních bezpilotních prostředků (UAV) se stal významným oborem mobilní robotiky.

Autonomní bezpilotní prostředky (UAV) stále častěji přecházejí z operací ve volném prostředí do komplexních prostředí, často bez signálu globálního navigačního satelitního systému (GNSS), což představuje významné výzvy pro koordinaci více robotů a autonomní navigaci jednoho robota. 
Tato práce se těmito problémy zabývá nejprve rozšířením algoritmu Rule-Based Lloyd (RBL) ze dvou do tří dimenzí pro koordinaci více UAV bez nutnosti komunikace. 
Byla vyvinuta nová pravidla pro řízení vertikálního pohybu a zlepšení efektivity 3D koordinace. 
Zadruhé, práce zkoumá praktické uplatnění tohoto 3D rozšíření pro navigaci jednoho UAV v náročném lesním prostředí s využitím palubního LiDAR sensoru.

Metodika zahrnovala přizpůsobení RBL algorithmu pro 3D prostor a jeho integraci se systémy vnímání v reálném čase. 
Pro úlohu navigace jednoho robota to zahrnovalo zpracování množiny bodů z LiDARu, odhad stavu pomocí Point-LIO estimace a voxelové mapování prostředí pomocí Bonxai. 
Byly také vyvinuty strategie pro minimalizaci omezení anizotropních senzorů, jako je omezené zorné pole (FOV) LiDARu. 
Navržený algoritmus byl evaluován pomocí rozsáhlých simulací s více roboty a validován experimenty v reálném lesním prostředí s jedním robotem.

Simulace potvrdily úspěšné 3D rozšíření algoritmu RBL a prokázaly efektivní koordinaci více robotů. 
Experimenty provedené v reálném prostředí potvrdily efektivitu systému založeném na 3D RBL pro autonomní navigaci z bodu do bodu v lese.
Byla však identifikována omezení, zejména ve způsobu, jakým mapovací systém zpracovával dynamické prvky prostředí, což občas vedlo k selhání navigace.



\vskip 1em

{\bf Klíčová slova} \KlicovaSlova

\vskip 2.5cm

\end{changemargin}
