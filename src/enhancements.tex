\chapter{Researching on possible enhancements of the algorithm for example using neural networks or learning-based techniques.\label{chap:enhancements}}

tune the paramaters of RBL to suit 3D

I will have to read something first, but what I found is that I could try object detection and classification using models like \href{https://arxiv.org/abs/1612.00593}{PointNet}, \href{https://arxiv.org/abs/1706.02413}{PointNet++} or \href{https://arxiv.org/abs/1711.06396}{VoxelNet}

Pointcloud denoising

Segmentation of LiDAR data.
\href{https://www.ipb.uni-bonn.de/wp-content/papercite-data/pdf/milioto2019iros.pdf}{RangeNet++}, \href{https://arxiv.org/abs/2003.03653}{SalsaNext} to differentiate between terrain trees and free space

NN approaches for approximating point clouds with simple convex shapes:

\href{https://openaccess.thecvf.com/content_CVPR_2020/papers/Deng_CvxNet_Learnable_Convex_Decomposition_CVPR_2020_paper.pdf}{CvxNet: Learnable Convex Decomposition by Deng et al. (CVPR 2020)}

\href{https://arxiv.org/abs/2003.13834}{Label-Efficient Learning on Point Clouds using Approximate Convex Decompositions by Gadelha et al. (Arxiv 2020)}

\href{https://arxiv.org/abs/1911.12870}{Region Segmentation via Deep Learning and Convex Optimization by Sonntag and Morgenshtern (Arxiv 2019)}

\href{https://arxiv.org/abs/2203.05662}{Point Density-Aware Voxels for LiDAR 3D Object Detection}