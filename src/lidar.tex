\chapter{Environment Perception Using 3D LiDAR\label{chap:lidar}}

\section{Introduction}

Motivation for using 3d lidar and challenges

\section{3D LiDAR Sensor Model and Simulation Setup}

Overview of LiDAR used. Simulation configuration

\section{Object Detection and Approximation}

Methods for extracting objects from LiDAR point clouds. Approximating detected objects with simple shapes. Handling noisy or incomplete data

% \href{https://people.cs.umass.edu/~smaji/papers/pcagan.pdf}{Shape Generation using Spatially Partitioned Point Clouds by Gadelha et al.}

\href{https://groups.inf.ed.ac.uk/advr/papers/3D_Surface_Approximation_from_Point_Clouds.pdf}{3D Surface Approximation from Point Clouds} $\leftarrow$ This one seems promising. I would like to try this.

% \href{https://par.nsf.gov/servlets/purl/10347732}{Label-Efficient Learning on Point Clouds using Approximate Convex Decompositions by Gadelha et al.}

\href{https://en.wikipedia.org/wiki/Alpha_shape}{Alpha Shape: A Generalization of the Convex Hull}



\section{Integration with RBL Algorithm}

Modifications to ensure safe navigation and how approximated objects influence Voronoi cells

\section{Simulation Results}

Example scenarios - in rviz with drone and also in real life - me holding a branch with leafs or something like that

\section{Discussion}

Limitations and possible improvements

\section{Summary}