\chapter{UAV Navigation Using the RBL Algorithm and LiDAR Sensing\label{chap:lidar}}

    \section{Introduction}
        TODO
        \subsection{Motivation}
            The importance of UAV navigation in environments with obstacles
        \subsection{Problem Statement}
            Challenges of UAV movement using LiDAR-based sensing
        \subsection{Objectives}
            Implementing RBL with LiDAR for obstacle avoidance and path planning
        \subsection{Chapter Overview}
            Summary of what this chapter covers

    \section{LiDAR-Based Perception and Point Cloud Processing}
        TODO
        \subsection{Overview of Lidar for UAV Navigation}
            role of lidar in uav perception and obstacel Detection
            advantages and challnges using lidar for realtime navigate
        \subsection{Point Cloud Data Acquisition}
            capturing spatial data from Lidar sensor - describe how the lidar works
            describe - ros topic that is being published and stuff
        \subsection{Preprocessing Techniques}
            Downsampling
            UAV Component filtering - removing points corresponding to uav structure - rods and blades
            Noise filtering
        \subsection{Surface Reconstruction and Triangulation}
            Greedy Projection Triangulation (GP3), pcl lib and parametres
            Generating triangular mesh - How GP3 constructs a surface from point cloud data
    
    \section{Using the found triangles for cell A partition}
        \subsection{Overview}
        the next step is to utilize these triangles to partition the navigation space, specifically modifying cell 
        A for the RBL algorithm. This process ensures that the UAV has a structured representation of obstacles and free space.
        \subsection{Finding the Closest Point on a Triangle}
        \subsection{Plane Calculation for Slicing Cell A}
        \subsection{Integration with the RBL Algorithm}

    \section{Implementation and Integration on UAV}
        \subsection{Software and Hardware setup}
            lidar sensor config, uav platform specifications
        \subsection{Software architecture}
            ros-based system designed
        \subsection{Challanges in Integration}
            livox - 360 * 60 deg - account for centroid calculation
    
    \section{Simulation and Experimental Results}
        \subsection{Simulation Setup}
            virtual environment modeling, tools used
        \subsection{Performance in Simulated Environments}
            Success rates, obstacle avoidance, efficiency metrics - same as in the rbl
        \subsection{Real-World Experiments in a Forest Environment}
        \subsection{Comparative Analysis}
            Simulation vs. real-world performance
    
    \section{Conclusion}
        Summary of contributions and some directions for future research
    






















REDO after this

\section{Introduction}

Motivation for using 3d lidar and challenges

\section{3D LiDAR Sensor Model and Simulation Setup}

Overview of LiDAR used. Simulation configuration

\section{Object Detection and Approximation}

Methods for extracting objects from LiDAR point clouds. Approximating detected objects with simple shapes. Handling noisy or incomplete data

% \href{https://people.cs.umass.edu/~smaji/papers/pcagan.pdf}{Shape Generation using Spatially Partitioned Point Clouds by Gadelha et al.}

\href{https://groups.inf.ed.ac.uk/advr/papers/3D_Surface_Approximation_from_Point_Clouds.pdf}{3D Surface Approximation from Point Clouds} $\leftarrow$ This one seems promising. I would like to try this.

% \href{https://par.nsf.gov/servlets/purl/10347732}{Label-Efficient Learning on Point Clouds using Approximate Convex Decompositions by Gadelha et al.}

\href{https://en.wikipedia.org/wiki/Alpha_shape}{Alpha Shape: A Generalization of the Convex Hull}



\section{Integration with RBL Algorithm}

Modifications to ensure safe navigation and how approximated objects influence Voronoi cells

\section{Simulation Results}

Example scenarios - in rviz with drone and also in real life - me holding a branch with leafs or something like that

\section{Discussion}

Limitations and possible improvements

\section{Summary}